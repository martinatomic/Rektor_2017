%%%%%%%%%%%%%%%%%%%%%%%%%%%%%%%%

\documentclass[11pt,a4paper]{article}
\usepackage{times}
\usepackage[utf8]{inputenc}
\usepackage[croatian]{babel}
\usepackage[T1]{fontenc} % Latin Modern

%%%%%%%%%%%%%%%%%%%%%%%%%%%%%%%%


%%%%%%%%%%%%%%%%%%%%%%%%%%%%%%%%
%%%%%%%%  MATEMATICKI PAKETI %%%%%%%%%%%
%%%%%%%%%%%%%%%%%%%%%%%%%%%%%%%%

\usepackage{amsmath}
\usepackage{amsfonts}
\usepackage{amssymb}
\usepackage{esvect}

%%%%%%%%%%%%%%%%%%%%%%%%%%%%%%%%

%%%%%%%%%%%%%%%%%%%%%%%%%%%%%%%%
%%%%%%%%%% PAKETI ZA SLIKE  %%%%%%%%%%%%
%%%%%%%%%%%%%%%%%%%%%%%%%%%%%%%%

\usepackage{graphicx}
\usepackage{float}
\usepackage[hidelinks]{hyperref}
\usepackage{caption}
\usepackage{subcaption}
\usepackage{booktabs}

%%%%%%%%%%%%%%%%%%%%%%%%%%%%%%%%

%%%%%%%%%%%%%%%%%%%%%%%%%%%%%%%%
%%%%%%%%%    PRORED 1.5   %%%%%%%%%%%%%%
%%%%%%%%%%%%%%%%%%%%%%%%%%%%%%%%

\renewcommand{\baselinestretch}{1.5}

%%%%%%%%%%%%%%%%%%%%%%%%%%%%%%%%


%%%%%%%%%%%%%%%%%%%%%%%%%%%%%%%%
%%%%%%%%%% TABLICA - ANTUN %%%%%%%%%%%%
%%%%%%%%%%%%%%%%%%%%%%%%%%%%%%%%

\usepackage{array}
\usepackage{multirow}
\newcolumntype{C}[1]{>{\centering\let\newline\\\arraybackslash\hspace{0pt}}m{#1}}
\newcolumntype{L}[1]{>{\raggedright\let\newline\\\arraybackslash\hspace{0pt}}m{#1}}
\newcolumntype{R}[1]{>{\raggedleft\let\newline\\\arraybackslash\hspace{0pt}}m{#1}}
\usepackage{ctable}

%%%%%%%%%%%%%%%%%%%%%%%%%%%%%%%%

%%%%%%%%%%%%%%%%%%%%%%%%%%%%%%%%
%%%%%%%%%% TABLICA - MARTINA %%%%%%%%%%%
%%%%%%%%%%%%%%%%%%%%%%%%%%%%%%%%

\makeatletter
\renewcommand*\env@matrix[1][\arraystretch]{%
  \edef\arraystretch{#1}%
  \hskip -\arraycolsep
  \let\@ifnextchar\new@ifnextchar
  \array{*\c@MaxMatrixCols c}}
\makeatother



%%%% LATEX KOD ZA KORISTENJE TABLICE %%%%
%%% PRIMJER %%%

%\setlength\extrarowheight{1pt}
%\begin{table}[h]
%\centering
%\caption{Tablica s prikazom }
%\label{prva}
%\begin{tabular}{|l|c|}
%\hline
%\textbf{txt} &  \\ \hline 
%txt & txt    \\ 
%txt & txt   \\ \hline
%txt & txt    \\ \hline
%\end{tabular}
%\end{table}

%%%%%%%%%%%%%%%%%%%%%%%%%%%%%%%%


%%%%%%%%%%%%%%%%%%%%%%%%%%%%%%%%
%%%%%%% DIO ZA UNOS ISJECAKA KODA %%%%%%%%
%%%%%%%%%%%%%%%%%%%%%%%%%%%%%%%%

\usepackage{listings}
\usepackage{color}
 
\definecolor{codegreen}{rgb}{0,0.6,0}
\definecolor{codegray}{rgb}{0.5,0.5,0.5}
\definecolor{codepurple}{rgb}{0.58,0,0.82}
 
\lstdefinestyle{mystyle}{   
    commentstyle=\color{codegreen},
    keywordstyle=\color{blue},
    numberstyle=\tiny\color{codegray},
    stringstyle=\color{codepurple},
    basicstyle=\footnotesize,
    breakatwhitespace=false,         
    breaklines=true,                 
    captionpos=b,                    
    keepspaces=true,                 
    numbers=left,                    
    numbersep=5pt,                  
    showspaces=false,                
    showstringspaces=false,
    showtabs=false,                  
    tabsize=1
}
 
\lstset{style=mystyle}

%\lstinputlisting[language=Matlab, firstline=1, lastline=4, numbers=left, frame=single, label={lst:prvi}, caption={Diskretizacija sustava korištenjem Matlaba}, captionpos=b]{peti.m} 

%%%%%%%%%%%%%%%%%%%%%%%%%%%%%%%%


%----------------------------
% za uredjenje stranice
\usepackage[left=2.5cm,right=2.5cm,top=2.5cm,bottom=2.5cm]{geometry}
\usepackage{fancyhdr}
\pagestyle{fancy} 
\lhead{\leftmark}
\rhead{\rightmark}
\usepackage{titlesec} %za točku iza broja sectiona
\titleformat{\section}{\huge\bfseries}{\thetitle.\quad}{0em}{}
\titleformat{\subsection}{\LARGE\bfseries}{\thetitle.\quad}{0em}{}
\titleformat{\subsubsection}{\Large\bfseries}{\thetitle.\quad}{0em}{}
\titleformat{\paragraph}
{\normalfont\large\bfseries}{\thetitle.\quad}{1em}{}
\titlespacing*{\paragraph}
{0pt}{3.25ex plus 1ex minus .2ex}{1.5ex plus .2ex}
\setcounter{secnumdepth}{5}

\usepackage{indentfirst} %uvlacenje prvog paragrafa
% primjer pozivanja sectiona
% \section*{UVOD} \pdfbookmark{UVOD}{section:UVOD}

\usepackage{tocloft}
\usepackage{import}
\usepackage{standalone}
\graphicspath{{figures/}} 

\hypersetup{
  colorlinks   = true, %Colours links instead of ugly boxes
  urlcolor     = black, %Colour for external hyperlinks
  linkcolor    = black, %Colour of internal links
  citecolor   = blue %Colour of citations
}

\usepackage{subcaption}
\usepackage{lscape}
\begin{document}

Robotika je grana tehnologije koja se bavi dizajnom, izradom, upravljanjem i primjenom robota. Ona je višedisciplinarna znanstvena grana koja objedinjuje znanja iz područja mehanike, elektronike, računarstva i automatike. \cite{kova} Robotika je istodobno vrlo privlačna, izazovna i maštovita disciplina. Zahvaljujući svakodnevnom napretku tehnologije, te novim postignućima u znanosti, robotika postaje neizostavan dio modernog društva. U posljednjem desetljeću, poseban fokus robotskog istraživanja stavljen je na zračnu robotiku, posebice na multirotorske letjelice. Od posebnog su interesa u istraživanju bespilotnih letjelica autonomne misije, navođenje i upravljanje, koordinacija multi-agentskih sustava letjelica, inovativni dizajn letjelica i drugo. 

\medskip

Potencijalne primjene bespilotnih letjelica dijele se na dvije grupe, vojne i civilne. Prva grupa, \textit{vojne aplikacije}, uključuje akcije kao što su target and decoy, praćenje i izviđanje, sudjelovanje u borbi i logističke primjene, dok druga grupa, \textit{civilne aplikacije}, uključuje akcije kao što su nadgledanje prometa i vremena, vatrogastvo, poljoprivreda, traženje i spašavanje, te istraživanje i razvoj. \cite{urs} Teži se razvoju autonomnih robotskih sustava, sposobnih za rad u izrazito dinamičnim i nedeterminističkim okruženjima.

\medskip

\begin{figure}[H]
	\centering
	\includegraphics[scale=0.35]{koord}
	\caption{Projekt MORUS - autonomni heterogeni robotski sustav sačinjen od bespilotne letjelice i bespilotne ronilice \cite{haus2}}
	\label{fig:koord}
\end{figure}

Ideja ovog rada proizlazi iz želje za razvojem jednog takvog autonomnog heterogenog robotskog sustava, sastavljenog od bespilotne letjelice i bespilotne ronilice, kao što je prikazano na slici \ref{fig:koord}. Ovakav robotski sustav služio bi u misijama izviđanja na moru. Znanstvenici sa Fakulteta elektrotehnike i računarstva u Zagrebu, na Zavodu za Automatiku i Računalno Inženjerstvo u Laboratoriju za Robotiku i Inteligentne Sustave Upravljanja, pod vodstvom prof. dr. sc. Stjepana Bogdana rade na razvoju jednog ovakvog sustava na projektu pod imenom MORUS. Cilj ovog robotskog sustava bila bi koordinirana akcija između letjelice i ronilice, u kojoj bi se ronilica u bazi na obali prikvačila za letjelicu, koja bi ju potom prevezla do mjesta na kojoj se obavlja određena misija te ju postavila u more i zatim se vratila u bazu. Po završetku misije letjelica bi otišla po ronilicu, izvukla ju iz mora te vratila na početni položaj, odnozno u bazu. \cite{haus2}


\medskip


Instinktivno je jasno kako su na ovakvu letjelicu postavljeni strogi uvjeti vezani za njezinu izdržljivost. Maksimalna masa tereta koju robot može prenijeti i maksimalna brzina gibanja robota ovise o tipu robota i njegovoj primjeni. \cite{kova}  Kako bi bespilotna letjelica bila sposobna podići i prenijeti bespilotnu ronilicu na predodređenu lokaciju, ona treba imati veliku nosivost ($\geq$ 50kg) i duže vrijeme leta ($\geq$ 30min) . Komercijalne bespilotne letjelice ne ispunjavaju zadane kriterije. Njihovo se upravljanje temelji na promjeni brzine rotacije propelera te na korištenju električnih DC motora. S obzirom na zahtjev veće nosivosti letjelice javila se potreba za razvojem u potpunosti novog sustava upravljanja. Električni DC motori su za početak zamjenjeni benzinskim motorima. Zbog spore dinamike benzinskih motora se uz klasični uvodi i novi koncept upravljanja temeljen na promjeni centra mase letjelice. \cite{haus1}

\medskip

Benzinski motori, unatoč svojim prednostima, otežavaju testiranje jednog ovakvog sustava. Za pokretanje kompleksnog sustava ovih razmjera potrebna je pomno planirana i vremenski dugotrajna priprema. Sami benzinski motori stvaraju buku i ispušne plinove te zahtijevaju pokretanje na otvorenom. U tu svrhu javlja se potreba za izradom laboratorijske makete letjelice. Glavna je motivacija biti u mogućnosti testirati algoritme upravljanja u laboratorijskim uvjetima. Ovaj rad stoga nastoji doprinijeti dizajnu i razvoju jedne takve letjelice te primijeniti i testirati novi koncept upravljanja bespilotnom letjelicom zasnovan na promjeni centra mase letjelice. 

\end{document}